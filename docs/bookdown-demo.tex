\documentclass[]{book}
\usepackage{lmodern}
\usepackage{amssymb,amsmath}
\usepackage{ifxetex,ifluatex}
\usepackage{fixltx2e} % provides \textsubscript
\ifnum 0\ifxetex 1\fi\ifluatex 1\fi=0 % if pdftex
  \usepackage[T1]{fontenc}
  \usepackage[utf8]{inputenc}
\else % if luatex or xelatex
  \ifxetex
    \usepackage{mathspec}
  \else
    \usepackage{fontspec}
  \fi
  \defaultfontfeatures{Ligatures=TeX,Scale=MatchLowercase}
\fi
% use upquote if available, for straight quotes in verbatim environments
\IfFileExists{upquote.sty}{\usepackage{upquote}}{}
% use microtype if available
\IfFileExists{microtype.sty}{%
\usepackage{microtype}
\UseMicrotypeSet[protrusion]{basicmath} % disable protrusion for tt fonts
}{}
\usepackage{hyperref}
\hypersetup{unicode=true,
            pdftitle={Seminário Latinoamericano: Instrumentos y metodologias para un observatório de Clima y su impacto en la salud humana},
            pdfauthor={Sergio Ibarra-Espinosa},
            pdfborder={0 0 0},
            breaklinks=true}
\urlstyle{same}  % don't use monospace font for urls
\usepackage{natbib}
\bibliographystyle{apalike}
\usepackage{color}
\usepackage{fancyvrb}
\newcommand{\VerbBar}{|}
\newcommand{\VERB}{\Verb[commandchars=\\\{\}]}
\DefineVerbatimEnvironment{Highlighting}{Verbatim}{commandchars=\\\{\}}
% Add ',fontsize=\small' for more characters per line
\usepackage{framed}
\definecolor{shadecolor}{RGB}{248,248,248}
\newenvironment{Shaded}{\begin{snugshade}}{\end{snugshade}}
\newcommand{\AlertTok}[1]{\textcolor[rgb]{0.94,0.16,0.16}{#1}}
\newcommand{\AnnotationTok}[1]{\textcolor[rgb]{0.56,0.35,0.01}{\textbf{\textit{#1}}}}
\newcommand{\AttributeTok}[1]{\textcolor[rgb]{0.77,0.63,0.00}{#1}}
\newcommand{\BaseNTok}[1]{\textcolor[rgb]{0.00,0.00,0.81}{#1}}
\newcommand{\BuiltInTok}[1]{#1}
\newcommand{\CharTok}[1]{\textcolor[rgb]{0.31,0.60,0.02}{#1}}
\newcommand{\CommentTok}[1]{\textcolor[rgb]{0.56,0.35,0.01}{\textit{#1}}}
\newcommand{\CommentVarTok}[1]{\textcolor[rgb]{0.56,0.35,0.01}{\textbf{\textit{#1}}}}
\newcommand{\ConstantTok}[1]{\textcolor[rgb]{0.00,0.00,0.00}{#1}}
\newcommand{\ControlFlowTok}[1]{\textcolor[rgb]{0.13,0.29,0.53}{\textbf{#1}}}
\newcommand{\DataTypeTok}[1]{\textcolor[rgb]{0.13,0.29,0.53}{#1}}
\newcommand{\DecValTok}[1]{\textcolor[rgb]{0.00,0.00,0.81}{#1}}
\newcommand{\DocumentationTok}[1]{\textcolor[rgb]{0.56,0.35,0.01}{\textbf{\textit{#1}}}}
\newcommand{\ErrorTok}[1]{\textcolor[rgb]{0.64,0.00,0.00}{\textbf{#1}}}
\newcommand{\ExtensionTok}[1]{#1}
\newcommand{\FloatTok}[1]{\textcolor[rgb]{0.00,0.00,0.81}{#1}}
\newcommand{\FunctionTok}[1]{\textcolor[rgb]{0.00,0.00,0.00}{#1}}
\newcommand{\ImportTok}[1]{#1}
\newcommand{\InformationTok}[1]{\textcolor[rgb]{0.56,0.35,0.01}{\textbf{\textit{#1}}}}
\newcommand{\KeywordTok}[1]{\textcolor[rgb]{0.13,0.29,0.53}{\textbf{#1}}}
\newcommand{\NormalTok}[1]{#1}
\newcommand{\OperatorTok}[1]{\textcolor[rgb]{0.81,0.36,0.00}{\textbf{#1}}}
\newcommand{\OtherTok}[1]{\textcolor[rgb]{0.56,0.35,0.01}{#1}}
\newcommand{\PreprocessorTok}[1]{\textcolor[rgb]{0.56,0.35,0.01}{\textit{#1}}}
\newcommand{\RegionMarkerTok}[1]{#1}
\newcommand{\SpecialCharTok}[1]{\textcolor[rgb]{0.00,0.00,0.00}{#1}}
\newcommand{\SpecialStringTok}[1]{\textcolor[rgb]{0.31,0.60,0.02}{#1}}
\newcommand{\StringTok}[1]{\textcolor[rgb]{0.31,0.60,0.02}{#1}}
\newcommand{\VariableTok}[1]{\textcolor[rgb]{0.00,0.00,0.00}{#1}}
\newcommand{\VerbatimStringTok}[1]{\textcolor[rgb]{0.31,0.60,0.02}{#1}}
\newcommand{\WarningTok}[1]{\textcolor[rgb]{0.56,0.35,0.01}{\textbf{\textit{#1}}}}
\usepackage{longtable,booktabs}
\usepackage{graphicx,grffile}
\makeatletter
\def\maxwidth{\ifdim\Gin@nat@width>\linewidth\linewidth\else\Gin@nat@width\fi}
\def\maxheight{\ifdim\Gin@nat@height>\textheight\textheight\else\Gin@nat@height\fi}
\makeatother
% Scale images if necessary, so that they will not overflow the page
% margins by default, and it is still possible to overwrite the defaults
% using explicit options in \includegraphics[width, height, ...]{}
\setkeys{Gin}{width=\maxwidth,height=\maxheight,keepaspectratio}
\IfFileExists{parskip.sty}{%
\usepackage{parskip}
}{% else
\setlength{\parindent}{0pt}
\setlength{\parskip}{6pt plus 2pt minus 1pt}
}
\setlength{\emergencystretch}{3em}  % prevent overfull lines
\providecommand{\tightlist}{%
  \setlength{\itemsep}{0pt}\setlength{\parskip}{0pt}}
\setcounter{secnumdepth}{5}
% Redefines (sub)paragraphs to behave more like sections
\ifx\paragraph\undefined\else
\let\oldparagraph\paragraph
\renewcommand{\paragraph}[1]{\oldparagraph{#1}\mbox{}}
\fi
\ifx\subparagraph\undefined\else
\let\oldsubparagraph\subparagraph
\renewcommand{\subparagraph}[1]{\oldsubparagraph{#1}\mbox{}}
\fi

%%% Use protect on footnotes to avoid problems with footnotes in titles
\let\rmarkdownfootnote\footnote%
\def\footnote{\protect\rmarkdownfootnote}

%%% Change title format to be more compact
\usepackage{titling}

% Create subtitle command for use in maketitle
\providecommand{\subtitle}[1]{
  \posttitle{
    \begin{center}\large#1\end{center}
    }
}

\setlength{\droptitle}{-2em}

  \title{Seminário Latinoamericano: Instrumentos y metodologias para un observatório de Clima y su impacto en la salud humana}
    \pretitle{\vspace{\droptitle}\centering\huge}
  \posttitle{\par}
    \author{Sergio Ibarra-Espinosa}
    \preauthor{\centering\large\emph}
  \postauthor{\par}
      \predate{\centering\large\emph}
  \postdate{\par}
    \date{2019-09-07}

\usepackage{booktabs}
\usepackage{amsthm}
\makeatletter
\def\thm@space@setup{%
  \thm@preskip=8pt plus 2pt minus 4pt
  \thm@postskip=\thm@preskip
}
\makeatother

\begin{document}
\maketitle

{
\setcounter{tocdepth}{1}
\tableofcontents
}
\hypertarget{curso-de-r-contaminacion-atmosferica-y-mas}{%
\chapter*{Curso de R, contaminacion atmosferica y mas}\label{curso-de-r-contaminacion-atmosferica-y-mas}}
\addcontentsline{toc}{chapter}{Curso de R, contaminacion atmosferica y mas}

Este curso online contendra las sisgueinetes informaciones

\begin{itemize}
\tightlist
\item
  Sistemas de informacion con dartos de salud en Chile (gracias Paty Matus)
\item
  Impacto de las emisiones antropogenicas en la salud y clima
\item
  R desde Excel
\item
  Leer y procesar vectores espaciales con \textbf{sf} \citep{sf}
\item
  Leer y procesar informacion en grillas espaciales (raster) con stars\citep{stars} y raster\citep{raster}
\end{itemize}

\hypertarget{aprender-git}{%
\section*{Aprender Git}\label{aprender-git}}
\addcontentsline{toc}{section}{Aprender Git}

Para aprender GIT puedes ver:

\begin{itemize}
\tightlist
\item
  \url{https://git-scm.com/book/es/v1/Empezando}
\item
  \url{https://learngitbranching.js.org/}
\item
  \url{https://try.github.io/}
\end{itemize}

\hypertarget{clonar-este-contenido}{%
\section*{Clonar este contenido}\label{clonar-este-contenido}}
\addcontentsline{toc}{section}{Clonar este contenido}

Para clonar este contenido haz:

\begin{Shaded}
\begin{Highlighting}[]
\FunctionTok{git}\NormalTok{ clone https://github.com/ibarraespinosa/UBA.git}
\end{Highlighting}
\end{Shaded}

\hypertarget{intro}{%
\chapter{Sistemas de informacion con datos de salud en Chile}\label{intro}}

\begin{itemize}
\tightlist
\item
  Sistema de información en salud existentes
\item
  Enfasis en las fuentes de información y las escala temporal/espacial que manejan
\item
  Series de tiempo disponible por fuente
\item
  Instituciones a cargo de la captura, procesamiento y análisis
\item
  Disponibilidad de los datos e indicadores que producen
\item
  Otros
\end{itemize}

\hypertarget{encuesta-nacional-de-salud-ens}{%
\section{\texorpdfstring{\href{http://www.encuestas.uc.cl/ens/index.html}{Encuesta Nacional de Salud (ENS)}}{Encuesta Nacional de Salud (ENS)}}\label{encuesta-nacional-de-salud-ens}}

La ENS es una encuesta realizada por el Ministerio de Salud para identificar cuales son las
enfermedades que sufren y los tratamientos que reciben todas las personas con mas de 15 años
que viven en Chile. De esta forma es posible es posible realizar diagnosticos,
identificar problemas y formular politicas planes y proyectos para mejor la salud de las personas.

\begin{itemize}
\tightlist
\item
  \emph{Organismo responsable}: Ministerio de Salud, Departamento de Epidemiología\\
  Gobierno de Chile.\\
\item
  \emph{Organismo ejecutor}: Pontificia Universidad Católica de Chile (PUC).\\
\item
  \emph{Población objetivo}: Personas de 15 años y más, chilenas o extranjeras que residen habitualmente en viviendas particulares ocupadas, localizadas en zonas urbanas y rurales de las quince regiones de Chile.\\
\item
  \emph{Representatividad}: Nacional, regional y Urbano/Rural.\\
\item
  \emph{Modo de aplicación}: Entrevista personal en hogar (Sistema de captura electrónica: Tablet), aplicada por encuestador y profesional enfermera de acuerdo al tipo de cuestionario.\\
\item
  \emph{Período de trabajo de campo}: Agosto 2016 a marzo 2017\\
\item
  \emph{Tamaño muestral}: 6.233 encuestados, de los cuales 5.520 cuentan con exámenes de laboratorio de acuerdo a protocolo. 37,1\% hombres, 62,9\% mujeres.\\
\item
  \emph{Error muestral}: Error absoluto de muestreo de 2,6\% a nivel nacional, raíz del efecto de diseño de 1,797, estimaciones con 95\% de confianza y error relativo inferior a 30\%.
\end{itemize}

Algunos resultados:

\begin{itemize}
\tightlist
\item
  Consumo de tabaco: 66,7\% no fuma, 33,\$ fuma.
\item
  Consumo riesgoso de alcohol 11,7\%, 20,5\% hombres, 3,3\% mujeres.
\item
  Sedentarismo: 86,7\%, 83,3\% hombre, 90.0\% mujeres.
\item
  Estado nutricional: 1,3\% enfraquecido, 24,5\% normal, 39,8\% sobrepeso, 31,2\% obeso, 3,2\% obeso morbido.
\item
  Sospecha de hipertension: 27,6\%.
\item
  Sospecha de diabetes: 12,3\%.
\item
  Autoreporte de infarto agudo al miocardio: 3,3\%.
\item
  Autoreporte de ataque cerebro vascular: 2,6\%.
\end{itemize}

Fuentes:

\begin{itemize}
\tightlist
\item
  \url{https://www.minsal.cl/wp-content/uploads/2017/11/ENS-2016-17_PRIMEROS-RESULTADOS.pdf}
\item
  \url{https://www.minsal.cl/wp-content/uploads/2018/01/2-Resultados-ENS_MINSAL_31_01_2018.pdf}
\item
  \url{http://www.encuestas.uc.cl/ens/index.html}
\end{itemize}

\hypertarget{departamento-de-estadisticas-e-informaciones-de-salud}{%
\section{\texorpdfstring{\href{http://www.deis.cl/}{Departamento de Estadisticas e Informaciones de Salud}}{Departamento de Estadisticas e Informaciones de Salud}}\label{departamento-de-estadisticas-e-informaciones-de-salud}}

\begin{itemize}
\tightlist
\item
  Resúmenes estadísticos mensuales (\href{http://www.deis.cl/bases-de-datos-rem/}{REM}). Vea el \href{http://estadisticas.ssosorno.cl/estadisticas/2017/manuales/2017-2018-Manual-Series-REM-V1.1.pdf}{manual}
\item
  \href{http://www.deis.cl/bases-de-datos-defunciones/}{Defunciones}
\item
  \href{http://www.deis.cl/descargar-bases-de-datos-2/?page_id=3487}{Egresos}
\item
  \href{http://www.deis.cl/descargar-bases-de-datos-2/?page_id=3493}{Nacimientos}
\item
  \href{http://www.deis.cl/descargar-bases-de-datos-2/?page_id=3499}{Atenciones de urgencia}
\item
  \href{http://www.deis.cl/descargar-bases-de-datos-2/?page_id=3784}{Enfermedades de notificacion obligatoria}
\item
  \href{http://www.deis.cl/estadisticas-eta/}{Enfermedades transmitidas por alimentos}
\item
  \href{http://www.deis.cl/?page_id=3946}{Tuberculosis}
\end{itemize}

\hypertarget{encuesta-de-caracterizacion-socioeconomica-casen}{%
\section{\texorpdfstring{\href{http://observatorio.ministeriodesarrollosocial.gob.cl/casen-multidimensional/casen/casen_2017.php}{Encuesta de caracterizacion socioeconomica (CASEN)}}{Encuesta de caracterizacion socioeconomica (CASEN)}}\label{encuesta-de-caracterizacion-socioeconomica-casen}}

``La Encuesta de Caracterización Socioeconómica Nacional (Casen) del Ministerio de Desarrollo Social es una encuesta a hogares, de carácter multipropósito, es decir, que abarca diversos temas como educación, trabajo, ingresos, salud, entre otros; además es una encuesta transversal, por lo tanto, incluye a todo el espectro de la población del país.''

\hypertarget{estadisticas-generales}{%
\section{Estadisticas generales}\label{estadisticas-generales}}

\begin{itemize}
\tightlist
\item
  \href{www.ine.cl}{Instituto Nacional de Estadisticas}
\end{itemize}

\hypertarget{cepal-stat}{%
\section{CEPAL STAT}\label{cepal-stat}}

\begin{itemize}
\tightlist
\item
  \href{http://estadisticas.cepal.org/cepalstat/WEB_CEPALSTAT/estadisticasIndicadores.asp}{Estadisticos e indicadores}
\item
  \href{http://estadisticas.cepal.org/cepalstat/WEB_CEPALSTAT/estadisticasIndicadores.asp}{Perfiles Nacionales}
\item
  \href{http://estadisticas.cepal.org/cepalstat/WEB_CEPALSTAT/PublicacionesEstadisticas.asp}{Publicaciones y estadisticas}
\end{itemize}

\hypertarget{banco-interamericano-de-desarrollo}{%
\section{Banco Interamericano de Desarrollo}\label{banco-interamericano-de-desarrollo}}

\begin{itemize}
\tightlist
\item
  \href{http://www.iadb.org/en/research-and-data//tables,6882.html?indicator=2}{Educacion}
\item
  \href{http://www.iadb.org/en/research-and-data//tables,6882.html?indicator=2}{Mercado Laboral}
\item
  \href{http://www.iadb.org/es/investigacion-y-datos//tablas,6882.html?indicator=4}{Ingreso}
\item
  \href{http://www.iadb.org/es/investigacion-y-datos//pobreza,7526.html}{Pobreza}
\item
  \href{http://www.iadb.org/es/investigacion-y-datos//tablas,6882.html?indicator=1}{Demografia}
\end{itemize}

Egresos hospitalarios 2001 -- 2016

Natalidad 2011

Mortalidad 1994 -- 2016

Casen 2009-2016

You can label chapter and section titles using \texttt{\{\#label\}} after them, e.g., we can reference Chapter \ref{intro}. If you do not manually label them, there will be automatic labels anyway, e.g., Chapter \ref{methods}.

Figures and tables with captions will be placed in \texttt{figure} and \texttt{table} environments, respectively.

\begin{Shaded}
\begin{Highlighting}[]
\KeywordTok{par}\NormalTok{(}\DataTypeTok{mar =} \KeywordTok{c}\NormalTok{(}\DecValTok{4}\NormalTok{, }\DecValTok{4}\NormalTok{, }\FloatTok{.1}\NormalTok{, }\FloatTok{.1}\NormalTok{))}
\KeywordTok{plot}\NormalTok{(pressure, }\DataTypeTok{type =} \StringTok{'b'}\NormalTok{, }\DataTypeTok{pch =} \DecValTok{19}\NormalTok{)}
\end{Highlighting}
\end{Shaded}

\begin{figure}

{\centering \includegraphics[width=0.8\linewidth]{bookdown-demo_files/figure-latex/nice-fig-1} 

}

\caption{Here is a nice figure!}\label{fig:nice-fig}
\end{figure}

Reference a figure by its code chunk label with the \texttt{fig:} prefix, e.g., see Figure \ref{fig:nice-fig}. Similarly, you can reference tables generated from \texttt{knitr::kable()}, e.g., see Table \ref{tab:nice-tab}.

\begin{Shaded}
\begin{Highlighting}[]
\NormalTok{knitr}\OperatorTok{::}\KeywordTok{kable}\NormalTok{(}
  \KeywordTok{head}\NormalTok{(iris, }\DecValTok{20}\NormalTok{), }\DataTypeTok{caption =} \StringTok{'Here is a nice table!'}\NormalTok{,}
  \DataTypeTok{booktabs =} \OtherTok{TRUE}
\NormalTok{)}
\end{Highlighting}
\end{Shaded}

\begin{table}[t]

\caption{\label{tab:nice-tab}Here is a nice table!}
\centering
\begin{tabular}{rrrrl}
\toprule
Sepal.Length & Sepal.Width & Petal.Length & Petal.Width & Species\\
\midrule
5.1 & 3.5 & 1.4 & 0.2 & setosa\\
4.9 & 3.0 & 1.4 & 0.2 & setosa\\
4.7 & 3.2 & 1.3 & 0.2 & setosa\\
4.6 & 3.1 & 1.5 & 0.2 & setosa\\
5.0 & 3.6 & 1.4 & 0.2 & setosa\\
\addlinespace
5.4 & 3.9 & 1.7 & 0.4 & setosa\\
4.6 & 3.4 & 1.4 & 0.3 & setosa\\
5.0 & 3.4 & 1.5 & 0.2 & setosa\\
4.4 & 2.9 & 1.4 & 0.2 & setosa\\
4.9 & 3.1 & 1.5 & 0.1 & setosa\\
\addlinespace
5.4 & 3.7 & 1.5 & 0.2 & setosa\\
4.8 & 3.4 & 1.6 & 0.2 & setosa\\
4.8 & 3.0 & 1.4 & 0.1 & setosa\\
4.3 & 3.0 & 1.1 & 0.1 & setosa\\
5.8 & 4.0 & 1.2 & 0.2 & setosa\\
\addlinespace
5.7 & 4.4 & 1.5 & 0.4 & setosa\\
5.4 & 3.9 & 1.3 & 0.4 & setosa\\
5.1 & 3.5 & 1.4 & 0.3 & setosa\\
5.7 & 3.8 & 1.7 & 0.3 & setosa\\
5.1 & 3.8 & 1.5 & 0.3 & setosa\\
\bottomrule
\end{tabular}
\end{table}

You can write citations, too. For example, we are using the \textbf{bookdown} package \citep{R-bookdown} in this sample book, which was built on top of R Markdown and \textbf{knitr} \citep{xie2015}.

\hypertarget{impacto-de-las-emisiones-antropogenicas-en-la-salud-y-clima}{%
\chapter{Impacto de las emisiones antropogenicas en la salud y clima}\label{impacto-de-las-emisiones-antropogenicas-en-la-salud-y-clima}}

\hypertarget{contaminacion-atmosferica}{%
\section{Contaminacion atmosferica}\label{contaminacion-atmosferica}}

La ciencia de la contaminacion atmosferica, si bien reciente, ha sido desarrollada debido a los avances de en la comprension
de la meteorologia.

Problemas relacionados con la contaminacion atmosferica han sido descritos en obras literarias y cartas a lo largo de
la historia. Por ejemplo, el primer caso reportado sobre los efectos de la contaminacion atmosférica en la salud es sobre Gaius Plinius Secundus (AD 23-AD 79), quien fallecio debido a los efectos de la \textbf{emisiones} del volcan Vesuvius \citep{art}.

{[}INSERTAR FOTO DE PLINIUS{]}

Sin embargo han sido los grandes episodios de contaminacion los que han gatillado su estudio y gestion por parte de los tomadores de decisiones. Entre ellos se pueden mensionar el desastre de Londres 1952 y la acidifacion de los lagos escandinavos.

\hypertarget{el-desastre-de-londres-1952}{%
\subsection{El desastre de Londres 1952}\label{el-desastre-de-londres-1952}}

{[}INSERTAR FOTO{]}

\hypertarget{acidicacion-de-los-gases-escandinavos}{%
\section{Acidicacion de los gases escandinavos}\label{acidicacion-de-los-gases-escandinavos}}

{[}INSERTAR FOTO{]}

\hypertarget{como-se-producen-las-altas-concentraciones-de-contaminantes}{%
\section{Como se producen las altas concentraciones de contaminantes}\label{como-se-producen-las-altas-concentraciones-de-contaminantes}}

\hypertarget{capa-de-mezcla}{%
\section{Capa de mezcla}\label{capa-de-mezcla}}

\hypertarget{section}{%
\section{}\label{section}}

\hypertarget{emisiones-y-sus-fuentes}{%
\section{Emisiones y sus fuentes}\label{emisiones-y-sus-fuentes}}

\hypertarget{efectos-de-la-contaminacion-atmosferica-en-la-salud}{%
\section{Efectos de la contaminacion atmosferica en la salud}\label{efectos-de-la-contaminacion-atmosferica-en-la-salud}}

\hypertarget{forzantes-climaticos}{%
\section{Forzantes climaticos}\label{forzantes-climaticos}}

\hypertarget{taller-vectores-aplicacion-de-software-de-informacion-geografica-y-modelado}{%
\chapter{Taller VECTORES: Aplicación de software de información geográfica y modelado}\label{taller-vectores-aplicacion-de-software-de-informacion-geografica-y-modelado}}

We describe our methods in this chapter.

\hypertarget{taller-raster-y-cubos-de-datos-vectoriales-aplicacion-de-software-de-informacion-geografica-y-modelado}{%
\chapter{Taller RASTER Y CUBOS DE DATOS VECTORIALES: Aplicación de software de información geográfica y modelado}\label{taller-raster-y-cubos-de-datos-vectoriales-aplicacion-de-software-de-informacion-geografica-y-modelado}}

Raster son informacion espaciales en una grilla espacial. Por ejemplo, vea las siguientes figuras:

\includegraphics{figs/cube1.png}

\includegraphics{figs/cube2.png}

Ejemplos con R

\hypertarget{final-words}{%
\chapter{Final Words}\label{final-words}}

We have finished a nice book.

\bibliography{book.bib,packages.bib}


\end{document}
